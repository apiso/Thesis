%\documentclass[12pt, preprint,numberedappendix]{emulateapj}
%\documentclass[12pt, preprint]{aastex}
%\documentclass[dvips,12pt]{article}
\documentclass[apj]{emulateapj}

\newcommand\submitms{n}		% set to y to follow AAS ``ms'' names, etc.
\newcommand\bibinc{n}		% set to y if bib pasted in .tex, set to n to use bibtex


%\usepackage{pdfsync}
\usepackage{subeqnarray}
\usepackage{natbib}
\usepackage{color}
\usepackage[utf8]{inputenc}

\bibliographystyle{apj}

\newcommand{\ie}{i.e.\ }
\newcommand{\eg}{e.g.\ }
\newcommand{\p}{\partial}
\newcommand{\brak}[1]{\langle #1\rangle}


\newcommand{\gcc}{\;\mathrm{g\; cm^{-3}}}
\newcommand{\gsc}{\;\mathrm{g\; cm^{-2}}}
\newcommand{\cm}{\; {\rm cm}}
\newcommand{\mm}{\; {\rm mm}}
%\newcommand{\ps}{\; {\rm s^{-1}}}
\newcommand{\km}{\; {\rm km}}
\newcommand{\au}{\; \varpi_{\rm AU}}
\newcommand{\AU}{\; {\rm AU}}
\def\K{\; {\rm K}}

\newcommand{\vcs}[1]{\mbox{\boldmath{$\scriptstyle{#1}$}}}
\newcommand{\vc}[1]{\mbox{\boldmath{$#1$}}}
\newcommand{\nab}{\vc{\nabla}}
\DeclareMathSymbol{\varOmega}{\mathord}{letters}{"0A}
\DeclareMathSymbol{\varSigma}{\mathord}{letters}{"06}
\DeclareMathSymbol{\varPsi}{\mathord}{letters}{"09}

\newcommand{\Eq}[1]{Equation\,(\ref{#1})}
\newcommand{\Eqs}[2]{Equations (\ref{#1}) and~(\ref{#2})}
\newcommand{\Eqss}[2]{Equations (\ref{#1})--(\ref{#2})}
\newcommand{\App}[1]{Appendix~\ref{#1}}
\newcommand{\Sec}[1]{Sect.~\ref{#1}}
\newcommand{\Chap}[1]{Chapter~\ref{#1}}
\newcommand{\Fig}[1]{Fig.~\ref{#1}}
\newcommand{\Figs}[2]{Figs.~\ref{#1} and \ref{#2}}
\newcommand{\Figss}[2]{Figs.~\ref{#1}--\ref{#2}} 
\newcommand{\Tab}[1]{Table \ref{#1}}

\definecolor{gray}{gray}{0.5}
\newcommand{\emgr}[1]{\emph{ \color{gray} #1}}


%\newenvironment{packed_item}{
%\begin{itemize}
%  \setlength{\itemsep}{1pt}
%  \setlength{\parskip}{0pt}
%  \setlength{\parsep}{0pt}
%}{\end{itemize}}

\begin{document}

%\slugcomment{Draft Modified \today}


\title{TBD (\textit{The role of ice compositions and morphology for snowlines the C/N/O ratios in active disks})}

\author{Ana-Maria A. Piso\altaffilmark{1}, Karin I. \"Oberg\altaffilmark{1}, et al. } %Tilman Birnstiel\altaffilmark{1}, Ruth A. Murray-Clay\altaffilmark{2}}
\altaffiltext{1}{Harvard-Smithsonian Center for Astrophysics, 60 Garden Street, Cambridge, MA 02138}
%\altaffiltext{2}{Department of Physics, University of California, Santa Barbara, CA 93106}


\begin{abstract}
...
\end{abstract}

\section{Introduction}

\emgr{Background info. Importance of volatiles in disks and planetary atmospheres, detections of snowlines in disks, C/O ratios etc. State again the importance of radial drift and gas accretion on the snowlines location, and that a systematic study of the combination of these two particular effects across the disk has not been done before. Then transition to the fact that we provide such a systematic study in Paper I and in this paper: here we expand the model of Paper I by making three additions: (1) we add N and CH4 in the static chemistry model, and explore how different abundances of CH4 and of the N main carriers (N2 and NH3) affect the C/O  and N/O ratios, (2) we quantify the effect of radial drift and gas accretion on N2, CH4 and NH3 snowline locations, and (3) we explore how different binding energies of CO, N2, affect the snowline locations.}

\section{Model Review}

\emgr{Review disk models, desorption model, relevant timescales. State that we use a steady-state disk for the coupled drift-desorption evolution, since it's the most realistic, therefore only summarize the static and steady-state disk. Summarize the findings of Paper I, i.e. particles of certain sizes desorb instantaneously and at a fixed particle size dependent location}

\section{CH4 and C/O Ratios}

\emgr{Discuss observed abundances for CH4 and the choices that we make (no CH4, median value, maximum value). Present new binding energies for CO as pure ice and mixed with water. Show Figure 1 and discuss how different CH4 abundances and binding energies affect snowline locations and C/O ratio: CO-H2O mixture (though I think it's rather CO layered on top of H2O) moves the CO snowline inward by $\sim$40 AU (will calculate percentages too); the maximum reasonable abundance of CH4 changes the C/O ratio by less than 10\%. Show Figure 2 and quantify the effect of drift and accretion on the CH4 snowline compared to a static disk. While CH4 has only a modest effect on the C/O ratio in a static disk, this effect may be larger in a viscous disk, as the C gas abundances inside the CH4 snowline may be enhanced due to the differential motion of the desorbed ices and overall nebular gas (refer to Paper I). In this study, however, we neglect these effects and therefore do not include CH4 in estimating the C/O ratio. Show Figure 2. Estimate how much }

\section{Nitrogen Carriers and N/O Ratios}

\emgr{Discuss that nitrogen is abundant in the solar system and disks and primarily found as N2. Due to the high volatility of N2, the gas phase N/O ratio in the outer disk may be even more enhanced than the C/O ratio. A fraction of the nitrogen abundance may be also carried by NH3. Discuss NH3 observed abundances and the choices that we make (no NH3, median, maximum). Show }

\section{Volatile Abundances and Binding Energies: Effect on C/O and N/O Ratios in a Static Disk}

\subsection{Nitrogen and CH4}

\emgr{Discuss that nitrogen is abundant in the solar system and should be abundant in disks as well, but that its dominant form is largely unknown. Discuss that the main carrier of nitrogen is N2, but some fraction of it can be NH3 as well. Present and motivate the choices that we make for NH3 abundances. Along the same lines, discuss CH4, and the choices that we make for CH4 abundances.}

\subsection{Volatile Desorption Energies}

\emgr{State that desorption energies for H2O and CO2 are well constrained experimentally, and that the CO2 binding energy is only weakly dependent on whether it's pure CO2 or combined with H2O, but that is not the case for CO, N2 (and perhaps CH4 and NH3?). Briefly discuss CO-CO, N2-N2, CO-H2O and N2-H2O (and perhaps the same for CH4 and NH3 if we find literature on that) experimental results for binding energies, and motivate the choices that we make.}

\subsection{Results for C/O and N/O in a Static Disk}

\emgr{For each of them (C/O and N/O), show a 3-panel plot as follows: each panel has a specific CH4/NH3 abundance (top: none, middle: median abundance, bottom: maximum abundance); for a given panel, have multiple curves for C/O or N/O, depending on the choice of binding energies, so that it's clear visually how the binding energy changes the snowline location. Discuss how different abundances  and binding energies affect snowline locations and C/O or N/O ratios.}
\section{Results}

\subsection{N2, NH3 and CH4 Snowline Locations}

\emgr{One multipanel 3x3 (or 3x2) rainbow plot similar to the snowline plots from Paper I, for \textit{one} choice for the binding energies (perhaps the largest ones, since we want a limit on how far in we can push the snowlines?). Rows: snowlines as a function or particle size for passive, active and (maybe) steady-state disk. Columns: N2, NH3 and CH4. Not entirely certain how necessary these plots/subsection actually are, since it is exactly the same qualitatively as in Paper I...}

\subsection{C/O and N/O Ratios}

\emgr{For each of them (C/O and N/O) show a 3x3 multipanel plot similar to the C/O plot from Paper I, for the same choice of binding energies as in the previous subsection. Columns: C/O or N/O in passive, active and steady-state disk. Rows: No CH4/NH3, median CH4/NH3, maximum CH4/NH3. Again, not sure if we need/want this for all disk choices, at least in the case of C/O since we already have that in Paper I. For N/O, quantify how the snowline location changes from a static disk due to drift and gas accretion.}

\section{Discussion}

\emgr{Discuss how entrapment of volatiles by H2O affects volatile abundances and C/O ratios. More TBD.}

\section{Summary}

\emgr{Maybe we can include the summary in the discussion section?}






\if\bibinc n
\bibliography{refs}
\fi

\if\bibinc y
\begin{thebibliography}
\end{thebibliography}
\fi


\end{document}