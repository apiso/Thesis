%\documentstyle[aas2pp4,epsf]{article}
%%%\documentstyle[aaspp4,epsf]{article}
%\documentstyle[12pt,aasms]{article}    % this is for a preprint
%(single-spaced)
%\documentstyle[aaspp4,epsf]{article} % this is for small print
%\documentstyle[12pt, aaspp4]{article}

%\documentstyle[11pt,aaspp]{article}
\documentclass[12pt, preprint]{aastex} 

%\documentclass[manuscript]{aastex}
%\documentclass[apj]{emulateapj}

%\documentclass[12pt, preprint,numberedappendix]{emulateapj}
%\documentstyle[12pt,aasms]{article}    % this is for submittal
                                       % (double-spaced)

%\documentstyle[12pt,aasms]{article}   \usepackage{emulateapj5} 

\usepackage{graphicx} 
\usepackage{amsmath}
\usepackage{hyperref}
\usepackage{amsfonts}
\usepackage{amsmath}
\usepackage{amssymb}
\usepackage{amsthm}
\usepackage{subeqnarray}
\usepackage{ulem}
%\bibliographystyle{apj}

\newcommand{\delad}{\nabla_{\rm ad}}
\newcommand{\delrad}{\nabla_{\rm rad}}
\newcommand{\emgr}[1]{\emph{ \color{gray} #1}}

\newcommand{\ie}{i.e.\ }
\newcommand{\eg}{e.g.\ }
\newcommand{\p}{\partial}
\newcommand{\xv}{\vc{x}}
\newcommand{\kv}{\vc{k}}
\newcommand{\brak}[1]{\langle #1\rangle}


\newcommand{\gcc}{\;\mathrm{g\; cm^{-3}}}
\newcommand{\gsc}{\;\mathrm{g\; cm^{-2}}}
\newcommand{\cm}{\; {\rm cm}}
\newcommand{\mm}{\; {\rm mm}}
%\newcommand{\ps}{\; {\rm s^{-1}}}
\newcommand{\km}{\; {\rm km}}
%\newcommand{\au}{\; \varpi_{\rm AU}}

\newcommand{\AU}{\; {\rm AU}}
\newcommand{\yr}{\; {\rm yr}}
\def\K{\; {\rm K}}

\newcommand{\vcs}[1]{\mbox{\boldmath{$\scriptstyle{#1}$}}}
\newcommand{\vc}[1]{\mbox{\boldmath{$#1$}}}
\newcommand{\nab}{\vc{\nabla}}
\DeclareMathSymbol{\varOmega}{\mathord}{letters}{"0A}
\DeclareMathSymbol{\varSigma}{\mathord}{letters}{"06}
\DeclareMathSymbol{\varPsi}{\mathord}{letters}{"09}

\newcommand{\Eq}[1]{Equation\,(\ref{#1})}
\newcommand{\Eqs}[2]{Equations (\ref{#1}) and~(\ref{#2})}
\newcommand{\Eqss}[2]{Equations (\ref{#1})--(\ref{#2})}
\newcommand{\Eqsss}[3]{Equations (\ref{#1}), (\ref{#2}) and~(\ref{#3})}
\newcommand{\App}[1]{Appendix~\ref{#1}}
\newcommand{\Sec}[1]{Sect.~\ref{#1}}
\newcommand{\Chap}[1]{Chapter~\ref{#1}}
\newcommand{\Fig}[1]{Fig.~\ref{#1}}
\newcommand{\Figs}[2]{Figs.~\ref{#1} and \ref{#2}}
\newcommand{\Figss}[2]{Figs.~\ref{#1}--\ref{#2}} 
\newcommand{\Tab}[1]{Table \ref{#1}}

\newenvironment{packed_item}{
\begin{itemize}
  \setlength{\itemsep}{1pt}
  \setlength{\parskip}{0pt}
  \setlength{\parsep}{0pt}
}{\end{itemize}}

%\newcommand{\delad}{\nabla_{\rm ad}}
%\newcommand{\delrad}{\nabla_{\rm rad}}
\newcommand{\Rg}{\mathcal{R}}
\newcommand{\RB}{R_{\rm B}}
\newcommand{\co}{_{\rm c}}
\newcommand{\di}{_{\rm d}}
\newcommand{\cb}{_{\rm RCB}}
\newcommand{\surf}{_M}
\newcommand{\mc}{m_{\rm c \oplus}}
\newcommand{\mcn}[1] { m_{ \rm c #1 \oplus} }
\newcommand{\MC}{M_{\rm crit}}
\newcommand{\au}{a_\oplus}
\newcommand{\aun}[1]{ a_{#1\oplus} }

\begin{document}
\bibliographystyle{apj}

\title{Giant Planet Atmospheres Do Not Have Stellar Composition}

\section{Processes that matter}

\begin{enumerate}
\addtocounter{enumi}{-1}
\item Main carriers of CNOH as a function of time
\item Which snow lines and where are they
\begin{itemize}
\item Assume nitrogen is mainly in $\rm N_2$ and $\rm NH_3$; assume hydrocarbons are primarily $\rm CH_4$ $\rightarrow$ three cases:
\begin{enumerate}
\item all nitrogen is $\rm N_2$ and no $\rm CH_4$
\item 10\% of nitrogen is $\rm NH_3$ and no $\rm CH_4$
\item all nitrogen is $\rm N_2$ and 5\% of carbon is in $\rm CH_4$
\end{enumerate}
%\item should also include HCN? If yes, start with Najita et al. (2013) on the HCN/$H_2O$ ratio
\item binding energy for $N_2$ = 790 $\pm$ 25 K \citep{oberg05}
\item $\rm N_2$ abundance: look in \citet{pontoppidan03}, papers cited in \citet{bisschop06}
\item $\rm NH_3$ abundance: \citet{lahuis00}, \citet{boogert08}, \citet{dodson09}
\item make C/O and N/O ratio plots similar to Figure 1 in \citet{oberg11}
\end{itemize}
\item Snow line evolution with time
\begin{itemize}
\item consider evolution both in terms of disk temperature and chemical composition
\item \citet{garaud07}, also papers by Scott Kenyon
\end{itemize}
\item Shape of snow line 
\begin{itemize}
\item include ``cold finger effect'', e.g. \citet{stevenson88}, \citet{cuzzi04}, \citet{ciesla06}
\item \citet{stevenson88} derive an approximate cold finger solution for the size of the area where the surface density of water ice is enhanced near the orbit of Jupiter
\end{itemize}
\item{Planetesimal build-up including drift}
\begin{itemize}
\item calculate drift rate as a function of the size of the planetesimal and compare to the evaporation rate; follow \citet{weidenschilling77}, \citet{chiang10} (which has all the correct coefficients) for the drift calculation
\item also calculate desorption timescale and the distance traveled by the planetesimal as a function of radius (e.g., for the $\rm H_2$O and CO snowlines)
\end{itemize}
\item Core accretion including migration
\begin{itemize}
\item can derive analytic formula similar to equation (2) from \citet{oberg11} that takes into account different C/O ratio between different snow lines
\item discuss both Type I and Type II migration
\end{itemize}
\item Runaway atmospheric accretion
\item{Late stage accretion}
\item{Core dredging}
\begin{itemize}
\item refs in \citet{lodders09}; also \citet{stevenson85}, \citet{guillot04}
\end{itemize}
\end{enumerate}

Other things to consider:

\begin{itemize}
\item Potential planet formation locations
\item Monte Carlo simulation including the processes that we find are relevant
\end{itemize}












%\begin{enumerate}
%\item Core migrates during accretion of planetesimals in the late stages of planet formation
%\begin{itemize}
%\item can derive analytic formula similar to equation (2) from Oberg et al. (2011) that takes into account different C/O ration between different snow lines
%\end{itemize}
%\item Include other molecules in simple analytic atmospheric model, i.e. nitrogen and hydrocarbons
%\begin{itemize}
%\item Assume nitrogen is mainly in $N_2$ and $NH_3$; assume hydrocarbons are primarily $CH_4$ $\rightarrow$ three cases:
%\begin{enumerate}
%\item all nitrogen is $N_2$ and no $CH_4$
%\item 10 \% of nitrogen is $NH_3$ and no $CH_4$
%\item all nitrogen is $N_2$ and 5\%(?) of carbon is in $CH_4$
%\end{enumerate}
%\item should also include HCN? If yes, start with Najita et al. (2013) on the HCN/$H_2O$ ratio
%\item binding energy for $N_2$ = 790 $\pm$ 25 K \citep{oberg05}
%\item $N_2$ abundance: look in Pontoppidan et al. (2003), papers cited in Bisschop et al. (2006)
%\item $NH_3$ abundance: Boogert et al. (2008), Dodson-Robinson et al. (2009, Icarus)
%\item $NH_3$ and HCN abundance: Lahuis \& van Dischoeck (2000)
%\item C/O ratio will only change if $CH_4$ (and potentially $HCN$) is included $\rightarrow$ make different plot from the Figure 1 from Oberg et al. (2011)?
%\end{itemize}
%\item Drift of solids across snow lines
%\begin{itemize}
%\item look at original derivation in Weidenschilling (1977) and try to derive an analytic estimate for length scale on which planetesimals evaporate after crossing a snow line?
%\end{itemize}
%\item Core dredging
%\begin{itemize}
%\item refs in Lodders et al. (2009); also Stevenson (1985), Guillot et al. (2004)
%\end{itemize}
%\item ``Cold finger effect''
%\begin{itemize}
%\item Stevenson \& Lunine (1988); also Cuzzi \& Zahnle (2004), Ciesla \& Cuzzi (2006)
%\item Stevenson \& Lunine (1988) derive an approximate cold finger solution for the size of the area where the surface density of water ice is enhanced near the orbit of Jupiter
%\end{itemize}
%\end{enumerate}


\bibliographystyle{apj}
\bibliography{refs}

\end{document}
