\chapter{Derivation of the Global Energy Equation}
\label{sec:globalderiv}

%\section{Derivation of the Global Energy Equation}\label{sec:globalderiv}

To derive  the global energy equation (\ref{eq:coolingglobal}) for an embedded protoplanet, we generalize the analogous calculations in stellar structure theory, e.g.\ in \S4.3 of \citet{kippenhahn90}.  For our problem, we add the effects of finite core radius, surface pressure and mass accretion. We start with the local energy equation (\ref{eq:structd}), whose more natural form in Lagrangian (mass) coordinates is $\p L/ \p m = \epsilon - T \p S /\p t$.  Integrating from the core to a higher shell with enclosed mass $M$ gives:
\begin{subeqnarray}
L - L\co &=& \int_{M\co}^M {\p L \over \p m} dm \\
&=& \int_{M\co}^M \left(\epsilon - T {\p S \over \p t} \right)dm \\
&=& \Gamma  - \int_{M\co}^M{\p u \over \p t} dm +  \int_{M\co}^M {P \over \rho^2} {\p \rho \over \p t} dm\slabel{eq:DLc}\, ,
\end{subeqnarray} 
with $\Gamma = \int \epsilon dm$ the integral of the direct heating rate, and applying the first law of thermodynamics in the final step.

The global energy equation is derived by eliminating the partial time derivatives in \Eq{eq:DLc}, which are performed at a fixed mass,
in favor of total time derivatives, denoted with overdots.  %The physical distinction is that total derivatives include mass  accreted through the outer boundary.  
For instance, the surface radius $R$ of the shell with enclosed mass $M$ evolves as  
\begin{equation}\label{eq:Rdot}
 \dot{R} = {\p R \over \p t} + {\dot{M} \over 4 \pi R^2 \rho_M},
\end{equation} 
where $\p R/\p t$ gives the Lagrangian contraction of the ``original" shell, and mass accretion through the upper boundary at rate $\dot{M}$ also changes the shell location.  
%The subscript $M$  denotes quantities at the upper boundary of total mass $M$ (though it is omitted from $M$ and $R$).  
Similarly, the volume $V = (4 \pi/3)R^3$ and pressure at the outer shell evolve as
\begin{subeqnarray}\label{eq:dot}
\dot{V}_M &=&  {\p V_{\rm M} \over \p t} + {\dot{M} \over \rho_{\rm M}}  \\
 \dot{P}_M &=& {\p P_{\rm M} \over \p t} + {\p P_M \over \p m}\dot{M} =  {\p P_{\rm M} \over \p t} - {G M  \over 4 \pi R^4} \dot{M}\, .
\end{subeqnarray} 
This derivation holds the core mass and radius fixed, $\dot{M}\co = \dot{R}\co = 0$.  Therefore, the core pressure satisfies
\begin{equation}\label{eq:Pcdot}
 \dot{P}\co = \p P\co / \p t \, .
\end{equation}
The internal energy integral follows simply from  Leibniz's rule as
\begin{equation}\label{eq:udot}
\int_{M\co}^{M(t)}{\p u \over \p t} dm = \dot{U}  -  \dot{M}u_M\, .
\end{equation} 
To make further progress, we use the virial theorem:
\begin{equation}
\label{eq:virial}
E_G=-3 \int_{M\co}^M \frac{P}{\rho} dm + 4 \pi (R^3 P_M-R\co^3 P\co),
\end{equation}
which follows from \Eqsss{eq:structb}{eq:structa}{eq:Eg} by integrating hydrostatic balance in Lagrangian coordinates.  As an aside, the integral in equation (\ref{eq:virial}) can be evaluated for a polytropic EOS to give simple expressions for the total energy:
\begin{subeqnarray}
E&=&(1-\zeta)U+4 \pi (R^3 P_M-R\co^3 P\co) \slabel{eq:vira} \\
&=&\frac{\zeta-1}{\zeta}E_G+\frac{4 \pi}{\zeta} (R^3 P_M-R\co^3 P\co) \slabel{eq:virb} \, ,
\end{subeqnarray}
where $\zeta \equiv 3(\gamma - 1)$.  We will not make this assumption and will keep the EOS general.

To express the work integral, i.e. the final term in \Eq{eq:DLc}, in terms of changes to gravitational energy, we first take the
 time derivative of \Eq{eq:virial}:
\begin{eqnarray}\label{eq:EGdot}
\dot{E}_G = 3  \int_{M\co}^M {P \over \rho^2} {\p \rho \over \p t} dm -3 \int_{M\co}^M {\p P\over \p t}{dm \over \rho} 
 -  3{P_M \over \rho_M} \dot{M}+ 3 \dot{P}_M V_M -3 \dot{P}\co V\co  + 3  P_M {\dot{ V}_M} \, . 
\end{eqnarray} 
%where the volumes, $V_M = 4 \pi R^3/3$ and $V\co = 4 \pi R\co^3/3$. 
The first integral in \Eq{eq:EGdot} is the one we want, but the second one must be eliminated.  The time derivative of \Eq{eq:Eg} (times four) gives
\begin{subeqnarray}
 4 \dot{E}_G &=&  -4 {G M \dot{M} \over R} + 4 \int_{M\co}^M {G m \over r^2}{\p r \over \p t} dm\\ 
&=&   -4 {G M \dot{M} \over R} + 4 \pi \int_{M\co}^M r^3{\p \over \p m}{\p P \over \p t} dm \slabel{eq:4EGb} \\
&=&  -4 {G M \dot{M} \over R} -3  \int_{M\co}^M {\p P\over \p t}{dm \over \rho}  + 3 V_M {\p P_M \over \p t} -3 V\co {\p P\co \over \p t} \slabel{eq:4EGc} \, ,
\end{subeqnarray} 
where \Eqs{eq:4EGb}{eq:4EGc} use hydrostatic balance  and integration by parts.

%To eliminate the time derivates of pressure, we take the time derivative of the hydrostatic balance equation for $\p^2 P / \p m\p t$ and integrate over $4\pi r^3 dm$ (as in the virial equation derivation) to get
%\begin{equation}\label{eq:dHBdt}
%3 \dot{P}_M V_M -3 \dot{P}\co V\co -3 \int_{M\co}^M {\p P\over \p t}{dm \over \rho}  = 4 \dot{E}_G + 4{G M \over R} \dot{M}  \, .
%\end{equation} 
%Combining \Eqs{eq:EGdot}{eq:dHBdt} gives 
%\begin{eqnarray}\label{eq:rhodot}
%\int_{M\co}^M {P \over \rho^2} {\p \rho \over \p t} dm  &=& - \dot{E}_G - {4 \over 3}{G M\over R} \dot{M} + {P_M \over \rho_M} \dot{M} -  P_M \dot{V}_M  \, , \nonumber \\
%&=&- \dot{E}_G - {4 \over 3}{G M\over R} \dot{M}  -  P_M {\p V_M \over \p t}  \, ,
%\end{eqnarray} 
%where the final step uses \Eq{eq:Rdot}.

Subtracting \Eqs{eq:udot}{eq:4EGc} and rearranging terms with the help of \Eqsss{eq:Rdot}{eq:dot}{eq:Pcdot} gives
\begin{eqnarray}\label{eq:PdVint}
\int_{M\co}^M {P \over \rho^2} {\p \rho \over \p t} dm  &=&  - \dot{E}_G - {G M \dot{M} \over R} - P_M {\p V_M \over \p t} \,  .
\end{eqnarray} 
Combining \Eqsss{eq:DLc}{eq:udot}{eq:PdVint}, we reproduce \Eq{eq:coolingglobal} with the accreted specific energy $e_M \equiv u_M - GM/R$.  

%\section{Analytic Cooling Model Details}\label{sec:analytic}
%TODO (academic): Figure out if theta < 1 is actually required? Should be but need to check
