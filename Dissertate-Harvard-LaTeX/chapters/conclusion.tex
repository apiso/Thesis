\chapter{Summary and Future Directions}
\label{conclusion}

In this thesis, I have calculated the minimum core mass required to form wide-separation gas giants at different disk locations, and for different equations of state of the nebular gas and dust opacities. I have also determined how disk dynamics and volatile abundance and morphology affect volatile snowline locations, the C/N/O ratios across the disk, and therefore the compositions of nascent giant planets. My results are summarized in Sections \ref{sec:core} and \ref{sec:volatile}, and I present directions for future work in Section \ref{sec:future}. 

\section{Minimum Core Masses for Giant Planet Formation}
\label{sec:core}

In Chapters 2 and 3, I determined the minimum core mass $M_{\rm crit}$ required to form a gas giant at distances between 5 and 100 AU from the host star, motivated by the fact that standard core accretion models cannot explain the formation of giant planets in the outer disk. This minimum applies when the solid cores are no longer accreting solids, since any additional heating due to planetesimal accretion would heat up the core's atmosphere, inhibit its ability to cool and contract, and therefore increase the critical core mass. I thus considered atmospheres accreting around fully formed cores and undergoing Kelvin-Helmholtz contraction. In Chapter 2, I developed a quasi-static atmospheric evolution model by generating a series of static atmospheric profiles that are then connected temporally through a cooling equation. I considered an envelope structure consisting of an inner convective region and an outer radiative layer, and assumed a constant luminosity in the outer radiative region. I demonstrated that this is a valid approximation for my parameter space of interest. For an ideal gas polytrope and standard interstellar (ISM) opacities, I have found that $M_{\rm crit}$ decreases with semimajor axis from $\sim 8.5 M_{\oplus}$ at 5 AU to $\sim 3.5 M_{\oplus}$ at 100 AU. My results are lower than the typically quoted value of $M_{\rm crit} \sim 10 M_{\oplus}$ (e.g., \citealt{rafikov06}), even in the more inner parts of the disk. 

To obtain more robust quantitative results, in Chapter 3 I expanded the model developed in Chapter 2 by considering a realistic equation of state (EOS) for the nebular gas and realistic dust opacities that take into account grain growth. I parametrized the EOS through the adiabatic gradient $\delad$ (defined in Chapter 2), as $\delad$ relates the gas pressure, temperature and density. While for an ideal gas EOS $\delad$ is constant, non-ideal EOS effects cause features in the adiabatic gradient that change the atmospheric structure. At high temperatures, molecular hydrogen dissociates, while at low-temperatures the rotational states of the hydrogen molecule are only partially excited so it no longer behaves like an ideal gas, due to the existence of H$_2$ in two spin isomeric forms, ortho- and parahydrogen. Both of these effects increase $M_{\rm crit}$ by a factor of $\sim$2 compared to the ideal gas. In contrast, grain growth opacities decrease $M_{\rm crit}$ dramatically. By taking these two competing effects together, I calculate $M_{\rm crit} \sim 8 M_{\oplus}$ at 5 AU, decreasing to $\sim 5 M_{\oplus}$ at 100 AU. While my atmospheric model is not equipped to calculate the critical core mass when grain coagulation is taken into account, I have demonstrated that grain growth with coagulation may decrease $M_{\rm crit}$ by up to one order of magnitude; $M_{\rm crit}$ may be as low as 1 $M_{\oplus}$. My study thus clearly challenges previous claims that core accretion cannot operate in the outer disk, reopening the case for in situ formation of wide separation gas giants. 

\section{The Role of Disk Dynamics and Ice Morphology on Snowline Locations and the C/N/O Ratios}
\label{sec:volatile}

In Chapters 4 and 5, I determined the effect of disk dynamics and ice morphology on the C/N/O ratio in active disks, which has direct implications on gas giant compositions. This study was motivated by the fact that the locations of volatile snowlines in protoplanetary disks are a defining feature of both gas giant and disk chemistry, as they provide vital information about the abundance of these molecules in gas and dust throughout the disk. In Chapter 4, I expanded the model of \citet{oberg11} by considering the effect of disk dynamics on the snowline locations of the main carbon and oxygen carriers, i.e. H$_2$O, CO$_2$ and CO, and thus on the C/O ratio in dust and gas throughout the disk. I studied the effect of radial drift of solids and viscous gas accretion onto the central star on snowline locations by developing a semi-analytical model, which I applied to three different disks in increasing order of complexity. I applied my model for a range of initial particle sizes and calculated how drift and gas accretion affect their desorption location. I have found that there is a range of particle sizes, $0.05$ cm $\lesssim s \lesssim$ 7 m for my particular parameters, that desorb almost instantaneously and at a fixed particle-size dependent location from the central star, regardless of their initial position. Based on this information, I calculated the H$_2$O, CO$_2$ and CO snowline locations for different particle sizes in this range, and made estimates for the C/O ratio throughout the disk. I have found that the snowlines move inward as the particle size increases, and may differ by up to a factor of $\sim$2 due to drift and gas accretion compared to a static disk, which does not experience any dynamical processes. This variation in snowline locations is significant, and has important consequences for the compositions of gas giants forming in situ. 

In Chapter 5, I expanded the model developed in Chapter 4 by considering additional volatiles, chemical abundances, and ice morphologies. As nitrogen is highly abundant in the Solar System and primarily found as N$_2$, I added nitrogen bearing species such as N$_2$ and NH$_3$ to my model, as well as hydrocarbons (specifically, CH$_4$). Motivated by laboratory experiments (e.g., \citealt{fayolle16}) that find significantly different binding energies for CO and N$_2$ depending on the ice environment in which they reside (pure versus water dominated ices), I calculated the CO and N$_2$ snowlines for both scenarios. I calculated the N/O ratio in static disks and found that it is highly enhanced compared to the stellar value: by a factor of $\sim$2 between the H$_2$O and CO$_2$ snowlines, by more than a factor of 3 between the CO$_2$ and CO snowlines, and by many orders of magnitude between the CO and N$_2$ snowlines, where oxygen gas is depleted. Thus I expect wide-separation giants to have an excess of nitrogen in their atmospheres, which may be used to trace their formation origins. I have also found that the binding environment has a large effect on the CO and N$_2$ snowline locations and may change them by factors of 3-4, both for static and dynamic disks. Thus by considering the combined effect of disk dynamics and ice morphology, the CO and N$_2$ snowline locations may change by up to a factor of $\sim$7, and may span 11-79 AU for N$_2$ in my disk model. This large uncertainty in snowline locations means that observations that anchor snowline locations at different stages of planet
formation are key to develop C/N/O ratios as a probe of planet formation zones. 

\section{Future Directions}
\label{sec:future}

My work has made important strides in our understudying of disk and planet compositions and the tight link between the two, as well as the effect of disk dynamics and chemistry on disk and planet compositions. As a first step, I plan to expand my work by considering the effect of diffusion on the C/N/O ratios in viscous steady-state disks. Similarly to the model of \citet{owen14}, I will develop a simplified method to estimate the abundance of different volatiles at various disk locations, and thus their effect on the C/N/O ratio both in static and dynamic disks. 

Through analytical and numerical calculations, I will explore a range of dynamical processes that may affect snowline locations and the distribution of volatiles in disks, expanding and generalizing the framework developed in Chapter 4 and 5. Such effects include particle growth and fragmentation, as well as variations in the gas mass accretion rate and stellar luminosity.

The complexity of disk chemistry means that coupling it with dynamical processes, while necessary, is non-trivial. As a next step I will thus couple the dynamical framework outlined above with time-dependent chemical models of increasing complexity, informed by results from state-of-the-art disk chemistry models (that can only be run on static disks).  By having a better understudying on how disk chemistry and dynamics affect the composition of nascent, and eventually mature planets, my work may provide essential context for characterizing the gas giants
that instruments such as JWST and the TESS will one day discover.  
 
