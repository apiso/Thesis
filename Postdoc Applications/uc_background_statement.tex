%\documentclass[12pt, preprint,numberedappendix]{emulateapj}
%\documentclass[12pt, preprint]{aastex}
%\documentclass[apj]{emulateapj}
\documentclass[12pt, letterpaper]{article}

%\newcommand\submitms{n}		% set to y to follow AAS ``ms'' names, etc.
%\newcommand\bibinc{n}		% set to y if bib pasted in .tex, set to n to use bibtex


%\usepackage{pdfsync}
%\usepackage{subeqnarray}
\usepackage[top=0.8in, bottom=0.7in, left=1in, right=1in]{geometry}
\usepackage{natbib}
\usepackage{color}
\usepackage{graphicx}
\usepackage{fancyhdr}
\usepackage[T1]{fontenc}
\usepackage{titling}
\usepackage{sectsty}
\usepackage{sidecap}
\usepackage{placeins}
\usepackage{indentfirst}
\setlength{\droptitle}{-7em}
\setlength{\abovecaptionskip}{-0.4ex}
\setlength{\belowcaptionskip}{-0.4ex}
%\pagenumbering{gobble}
\pagestyle{fancy}
\lhead{Ana-Maria Piso}
\rhead{Education and Background Statement}
\date{}
%\rhead{\thepage}


%\bibliographystyle{apj}

\title{\Large UC President's Postdoctoral Fellowship Education and Background Statement}
\author{Dr. Ana-Maria Piso}

%\newenvironment{packed_item}{
%\begin{itemize}
%  \setlength{\itemsep}{1pt}
%  \setlength{\parskip}{0pt}
%  \setlength{\parsep}{0pt}
%}{\end{itemize}}

\begin{document}
\maketitle

%\slugcomment{Draft Modified \today}

\vspace{-0.9cm}

I was born in Romania a few years before the 1989 revolution. As a woman growing up in a country that was not ready to promote values such as diversity, inclusion, tolerance, and equality between genders, I was not expected to pursue a scientific career. My family supported my choice, but society and the educational system did not. I moved to the United States to study Physics at MIT with the expectation that the values and ethics that are often disregarded in my home country would be universally appreciated here. Unfortunately, I found out that this is not always the case, especially regarding gender equality. As I gradually advanced in my career in astrophysics, this disparity became more and more evident. As did the fact that girls and young women are often discouraged from considering scientific career paths, either due to the scarcity of opportunities and resources, or their own lack of self confidence, galvanized by little societal support or very few role models. I am therefore highly dedicated to mentoring, encouraging and empowering women, a group often deeply underrepresented in scientific fields.  

Throughout my graduate studies at Harvard University, I have often been a mentor, in various capacities, to younger women already pursuing a scientific career, or just interested in and excited about the possibility. I was part of the Women In Science, Engineering, Technology and Mathematics (WISTEM) organization, a group that aims to foster a sense of community for women studying science and engineering at Harvard. The program pairs graduate and undergraduate students in a mentor-mentee relationship, with the aim of creating a network for women scientists. I served as a mentor for three years to a Harvard undergraduate. Through monthly coffees or lunches, we discussed graduate school opportunities, work/life balance, some of the challenges that she faced both as a woman and a minority, and the general prospect of a career in science. She had successful summer internships at Harvard Medical School and Harvard School of Public Health, and is now in her final year as a biostatistics major. In a similar capacity, I have been a mentor for a summer Research Experiences for Undergraduates (REU) student at the Center for Astrophysics (CfA). As part of Prof. Karin \"Oberg's group, I co-mentored two of her female undergraduate students for their senior thesis projects, one of whom received the award for best senior thesis, and the other who is now a graduate student at the CfA.   

I am also committed to encouraging and empowering girls from minority groups or underprivileged communities. I was a Mentor Scientist at the Science Club For Girls (SCFG), a forum in which undergraduate and graduate women students in STEM fields teach elementary school girls from challenging social backgrounds some basic scientific concepts in a fun, interactive way. The goal is twofold --- to stimulate the girls' interest in science from a young age, and to empower them by serving as female role models. The format of the classes was usually a combination of discussion and hands-on experiments --- for example, second grade girls had to
make a prism and shine a flashlight through it to see the forming rainbow, and fourth grade girls had to filter dirty water using sand and cotton filters. As a SCFG Mentor Scientist, I learned how to explain very basic scientific concepts in a manner that is intuitive and understandable for a young audience, as well as resist the urge to jump in to help the girls in their experiments, but rather 
let them figure things out by themselves and thus boost their confidence. 

At UCLA, I plan to join the Sociable Postdoc Society, a group that aims to create a community between postdoctoral scholars from different UCLA departments, as well as between postdocs at other local campuses. Among other things, I would like to organize regular social/academic lunches, in which scholars from different departments would give a short, widely accessible presentation about their research, while also having the chance to socialize and network with other postdocs. I believe this would be a great opportunity for people with similar career paths, and yet with different backgrounds, genders, nationalities, ethnicities, to interact, learn from each other's experiences, and embrace their diversity.



%\end{thebibliography}

%\bibliographystyle{abbrv}
%\bibliography{refs}


%\if\bibinc y
%\begin{thebibliography}
%\end{thebibliography}
%\fi


\end{document}