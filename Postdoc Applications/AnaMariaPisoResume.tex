\documentclass[margin,line]{res}
\usepackage{hyperref}


\oddsidemargin -.5in
\evensidemargin -.5in
\textwidth=6.0in
\itemsep=0in
\parsep=0in

\newenvironment{list1}{
  \begin{list}{\ding{113}}{%
      \setlength{\itemsep}{0in}
      \setlength{\parsep}{0in} \setlength{\parskip}{0in}
      \setlength{\topsep}{0in} \setlength{\partopsep}{0in} 
      \setlength{\leftmargin}{0.17in}}}{\end{list}}
\newenvironment{list2}{
  \begin{list}{$\bullet$}{%
      \setlength{\itemsep}{0in}
      \setlength{\parsep}{0in} \setlength{\parskip}{0in}
      \setlength{\topsep}{0in} \setlength{\partopsep}{0in} 
      \setlength{\leftmargin}{0.2in}}}{\end{list}}


\begin{document}

\name{Ana-Maria A. Piso \vspace*{.1in}}

\begin{resume}
\section{\sc Contact Information}
\vspace{.05in}
\begin{tabular}{@{}p{3.5in}p{4in}}
           
Harvard-Smithsonian Center for Astrophysics & {\it Phone:}    (617) 818-6780 \\         
60 Garden Street, MS-10 & {\it E-mail:}  apiso@cfa.harvard.edu\\       
Cambridge, MA  02138  & {\it WWW:} www.cfa.harvard.edu/$\sim$apiso \\     
\end{tabular}


%\section{\sc Research Interests}
%Bayesian statistics, spatial statistics, nonparametric regression,
%statistical methods for large datasets, statistics for public policy

\section{\sc Education}
{\bf Harvard University}, Cambridge, MA \\
\begin{list1}
\vspace{-0.1in}
\item[] Ph.D., Astronomy \& Astrophysics, May 2016 (expected)
\item[] Advisor: Dr. Karin \"Oberg 
\item[] Thesis Topic: ``The Effect of Disk Chemistry and Dynamics in Shaping Volatile Snowlines''   
\end{list1}
{\bf Harvard University}, Cambridge, MA \\
\begin{list1}
\vspace{-0.1in}
\item[] A.M., Astronomy \& Astrophysics, May 2013
\item[] Advisor: Dr. Ruth Murray-Clay
\item[] Research Exam Project: ``On the Minimum Core Mass for Giant Planet Formation''
\end{list1}
{\bf Massachusetts Institute of Technology}, Cambridge, MA\\
%{\em Department of Statistics} 
\vspace*{-.1in}
\begin{list1}
\item[] S.B., Physics,  June 2010 \,\,\,\,\,\,\,\,\,\,\,\,\,\,\,\, Major GPA: 4.6/5.0
\item[] S.B., Mathematics, June 2010\,\,\,\, Major GPA: 4.8/5.0
\end{list1}

\section{\sc Research Experience \& Employment}
\begin{list2}
\item[] {\bf Research assistant} \hfill {\bf August 2010 - July 2011}
%{\bf Harvard-Smithsonian Center for Astrophysics} \hfill \textit{Cambridge, MA}
\begin{list1}
%\vspace{-0.1in}
\item[] MIT, EAPS Department \hfill \textit{Cambridge, MA}
\item[] Project: The Magnetic Field Signature of Super Earths
\item[] Advisor: Prof. Sara Seager \\
\end{list1}

\item[] {\bf Undergraduate researcher} \hfill {\bf June 2008 - June 2010}
%{\bf Harvard-Smithsonian Center for Astrophysics} \hfill \textit{Cambridge, MA}
\begin{list1}
%\vspace{-0.1in}
\item[] MIT, Kavli Institute or Astrophysics \hfill \textit{Cambridge, MA}
\item[] Project: The Solar Wind (2008) \& Structure of Accretion Disks (2009 - 2010)
\item[] Advisors: Dr. Paola Rebusco \& Prof. Edmund Bertschinger \\
\end{list1}

\item[] {\bf Research assistant} \hfill {\bf June 2009 - August 2009}
%{\bf Harvard-Smithsonian Center for Astrophysics} \hfill \textit{Cambridge, MA}
\begin{list1}
%\vspace{-0.1in}
\item[] Vienna University of Technology (TU Wien) \hfill \textit{Vienna, Austria}
\item[] Project: Exact relativistic viscous fluid solutions in near horizon extremal Kerr background
\item[] Advisor: Dr. Daniel Grumiller \\
\end{list1}

\item[] {\bf Undergraduate researcher} \hfill {\bf January 2007 - August 2007}
%{\bf Harvard-Smithsonian Center for Astrophysics} \hfill \textit{Cambridge, MA}
\begin{list1}
%\vspace{-0.1in}
\item[] MIT, Laboratory of Nuclear Science \hfill \textit{Cambridge, MA}
\item[] Project: Dark Matter Direct Detection
\item[] Advisors: Prof. Gabriela Sciolla \& Dr. Denis Dujmic \\
\end{list1}

\item[] {\bf Assistant manager} \hfill {\bf November 2005 - June 2006}
%{\bf Harvard-Smithsonian Center for Astrophysics} \hfill \textit{Cambridge, MA}
\begin{list1}
%\vspace{-0.1in}
\item[] Neuron Group S.R.L. Software Company \hfill \textit{Bucharest, Romania}
\item[] Digital map designer and database manager for the '112 Emergency Call Center' national project
%\item[] Advisors: Dr. Paola Rebusco \& Prof. Edmund Bertschinger
\end{list1}
\end{list2}

%\vspace{-.4cm}
%\textbf {\emph{Research Assistant}} \hfill {\bf August 2011 - present}\\  
%\vspace*{-.17in}
%\begin{list2}
%\item Conducted research in the area of exoplanetary magnetic fields
%\item Modeled the planet---star magnetic interaction between super Earths and their host stars
%\item Submitted results for publication as a first author to the Astrophysical Journal
%\end{list2}
%
%{\bf MIT, EAPS Department} \hfill \textit{Cambridge, MA}
%
%\vspace{-.4cm}
%\textbf {\emph{Research Assistant}} \hfill {\bf August 2010 - July 2011}\\  
%\vspace*{-.17in}
%\begin{list2}
%\item Conducted research in the area of exoplanetary magnetic fields
%\item Modeled the planet---star magnetic interaction between super Earths and their host stars
%\item Submitted results for publication as a first author to the Astrophysical Journal
%\end{list2}
%
%{\bf Vienna University of Technology (TU Wien)} \hfill \textit{Vienna, Austria}
%
%\vspace{-.4cm}
%\textbf {\emph{Researcher}} \hfill {\bf June 2009 - August 2009}\\  
%\vspace*{-.17in}
%\begin{list2}
%\item Conducted supervised research on extremal Kerr black holes 
%\item Presented on solar wind at a seminar aimed at graduate students and postdoctoral fellows
%\item Submitted results for publication in Physical Review D journal
%\end{list2}
%
%{\bf MIT, Kavli Institute for Astrophysics} \hfill \textit{Cambridge, MA}
%
%\vspace{-.4cm}
%\textbf {\emph{Researcher in the UROP \footnote{Undegraduate Research Opportunities Program}}} \hfill {\bf June 2008 - June 2010}\\  
%\vspace*{-.17in}
%\begin{list2}
%\item  Conducted supervised research in the fields of solar wind and accretion disks
%\item Performed simulations using the Mathematica software
%\item Designed and published two Mathematica Demonstration Projects
%\end{list2}
%
%{\bf MIT, Laboratory of Nuclear Science} \hfill \textit{Cambridge, MA}
%
%\vspace{-.4cm}
%\textbf {\emph{Researcher in the UROP}} \hfill {\bf January 2007 - August 2007}\\  
%\vspace*{-.17in}
%\begin{list2}
%\item Collaborated with a group of five researchers on directional detection of dark matter 
%\item Collected experimental data from a CCD camera and analyzed it using the ROOT framework 
%\item Built flash animations for presentations and updated the LNS website
%\end{list2}
%
%{\bf NEURON GROUP S.R.L. Software Company} \hfill \textit{Bucharest, Romania}
%
%\vspace{-.4cm}
%\textbf {\emph{Assistant Manager}} \hfill {\bf November 2005 - June 2006}\\  
%\vspace*{-.17in}
%\begin{list2}
%\item Worked on the '112 Emergency Call Center' national project 
%\item Created digital maps and associated databases for Romanian cities and counties
%\item Prepared analyses and reviews of the company's projects for several conferences and fairs
%\end{list2}

\section{\sc Teaching \& Outreach}
\begin{list2}
\item[] {\bf Teaching Fellow} \hfill {\bf February 2012 - May 2012}
%{\bf Harvard-Smithsonian Center for Astrophysics} \hfill \textit{Cambridge, MA}
\begin{list1}
%\vspace{-0.1in}
\item[] Harvard College class SPU 30: Life as a Planetary Phenomenon \hfill \textit{Cambridge, MA}
\item[] Course Head: Prof. Dimitar Sasselov
\item[] Held two weekly two-hour sections
\end{list1}

\vspace{0.2in}

\item[] {\bf Science Club For Girls Mentor Scientist} \hfill { \bf September 2014 - May 2015}
\begin{list1}
\item[] Taught second grade girls at the Amigos School the class ``Sound \& Light'' \hfill \textit{Cambridge, MA}
\end{list1}

\vspace{0.2in}

\item[] {\bf WISTEM Program Mentor} \hfill {\bf September 2013 - present}
\begin{list1}
\item[] Mentor for a Harvard College undergraduate \hfill \textit{Cambridge, MA}
\end{list1}

\vspace{0.2in}

\item[] {\bf CfA Summer Mentor} \hfill {\bf June 2014 - August 2014}
\begin{list1}
\item[] Co-mentored an REU summer student \hfill \textit{Cambridge, MA}
\end{list1}

\vspace{0.2in}

\item[] {\bf Co-Organizer of Harvard Graduate Student Prospective Visits} \hfill {\bf March 2013}
\begin{list1}
\item[] Organized and coordinated meetings and activities for two groups of ~10 prospective graduate students each \hfill \textit{Cambridge, MA}
\end{list1}

\end{list2}

%
%{\bf Harvard-Smithsonian Center for Astrophysics} \hfill \textit{Cambridge, MA}
%
%\vspace{-.4cm}
%\textbf {\emph{Teaching Fellow}} \hfill {\bf February 2012 - May 2012}\\  
%\vspace*{-.17in}
%\begin{list2}
%\item Teaching assistant for SPU 30: Life as a Planetary Phenomenon (Course Head: Dimitar Sasselov)
%\item Held two weekly two-hour sections
%\item Led discussions, participated in TF meetings and graded weekly problem sets
%\end{list2}

\section{\sc Refereed Publications}

\begin{list2}
\item[] {\bf Piso, A.-M. A.}, \"Oberg, K. I., Birnstiel, T., \& Murray-Clay, R. A. \textit{C/O and Snowline Locations in Protoplanetary Disks: The Effect of Radial Drift and Viscous Gas Accretion}. ApJ, under review \\
\end{list2}

\begin{list2}
\item[] {\bf Piso, A.-M. A.}, Youdin, A. N., \& Murray-Clay, R. A. \textit{Minimum Core Masses for Giant Planet Formation with Realistic Equations of State and Opacities}. ApJ, 2015, 800, 82 \\
\end{list2}

\begin{list2}
\item[] {\bf Piso, A.-M. A.} \& Youdin, A. N. \textit{On the Minimum Core Mass for Giant Planet Formation at Wide Separations}. ApJ, 2014, 786, 21 \\
\end{list2}
%\end{list1}

\section{\sc Publications in Preparation}


\begin{list2}
\item[] {\bf Piso, A.-M. A.}, \"Oberg, K.I., \& Pegues, J. \textit{The Role of Ice Compositions and Morphology For Snowlines and the C/N/O Ratios in Active Disks} \\
\end{list2}

%\begin{list1}
%\item[] {\bf On the Minimum Core Mass for Giant Planet Formation} (2013, in prep.) \\
%(http://web.mit.edu/apiso/www/MagFieldFinal.pdf) - submitted to the Astrophysical Journal\\
%Authors: Ana-Maria A. Piso and Andrew N. Youdin  \\
%\end{list1}



%\begin{list2}
%\item {\bf Magnetic field signature of Super Earths} \\
%(http://web.mit.edu/apiso/www/MagFieldFinal.pdf) - submitted to the Astrophysical Journal\\
%Authors: Ana-Maria Piso, Paola Rebusco and Sara Seager\\
%\end{list2}
%
%\vspace*{-.13in}
%\begin{list2}
%\item {\bf Exact relativistic viscous fluid solutions in near horizon extremal Kerr background} \\
%(arxiv.org: 0909.2041 [astro-ph.SR]) - submitted to Physical Review D\\
%Authors: Daniel Grumiller and Ana-Maria Piso\\
%\end{list2}

%\vspace{0.2in}

\section{\sc Online Publications \& Educational Material}

%\vspace*{-.13in}
\begin{list2}
\item[] {\bf The Solar Wind} \\
(Mathematica Demonstration Project: http://demonstrations.wolfram.com/TheSolarWind/) \\
Author: Ana-Maria Piso \\
\end{list2}

\vspace*{-.13in}
\begin{list2}
\item[] {\bf The Interplanetary Magnetic Field (Parker Spiral)} \\
(Mathematica Demonstration Project: \\
http://demonstrations.wolfram.com/TheInterplanetaryMagneticFieldParkerSpiral/ \\
Author: Ana-Maria Piso \\
\end{list2}


\section{\sc Conferences and Seminars}

\begin{list2}
\item[] {\bf Minimum Core Masses for Giant Planet Formation} \\
CfA Exoplanet Pizza Lunch, Cambridge, MA, May 2015  \\
Internal department talk \\
\end{list2}

\begin{list2}
\item[] {\bf Minimum Core Masses for Giant Planet Formation} \\
Star and Planet Formation in the Southwest, Oracle, AZ, March 2015  \\
Contributed talk \\
\end{list2}

\begin{list2}
\item[] {\bf On the Minimum Core Mass for Giant Planet Formation} \\
CfA Exoplanet Pizza Lunch, Cambridge, MA, November 2013  \\
Internal department talk \\
\end{list2}

\begin{list2}
\item[] {\bf On the Minimum Core Mass for Giant Planet Formation} \\
Protostars and Planets VI, Heidelberg, Germany, July 2013  \\
Poster \\
\end{list2}

\begin{list2}
\item[] {\bf On the Minimum Core Mass for Giant Planet Formation} \\
IAUS 299: Exploring the Formation and Evolution of Planetary Systems, Victoria, BC, June 2013  \\
Contributed talk \\
\end{list2}

\begin{list2}
\item[] {\bf The Structure and Stability of Atmospheres Accreting around Protoplanetary Cores} \\
Exoplanets in Multi-body Systems in the Kepler Era, Aspen, CO, February 2013 \\
Poster \\
\end{list2}

\begin{list2}
\item[] {\bf Magnetic field signature of Super Earths} \\
AAS 217$^{th}$ Meeting, Washington, Seattle, January 2011  \\
Poster \\
\end{list2}

\vspace*{-.13in}
\begin{list2}
\item[] {\bf Exact relativistic viscous fluid solutions in NHEK background} \\
APS April Meeting, Washington, DC, February 2010  \\
Poster \\
\end{list2}

\vspace*{-.13in}
\begin{list2}
\item[] {\bf The Solar Wind}\\ 
Vienna Theory Lunch Club, TU Wien, Vienna, Austria, June 2009  \\
Invited talk \\
\end{list2}

%\section{\sc Leadership}
%{\bf Delta Psi Co-ed Fraternity} \hfill \textit{Cambridge, MA}
%
%\vspace{-.4cm}
%\textbf {\emph{President/House Manager}} \hfill {\bf February 2008 - February 2009}\\  
%\vspace*{-.17in}
%\begin{list2}
%\item Coordinated a 5-day work week at the beginning of the fall semester
%\item Supervised the assembling of wooden floors in the house library and acquired furniture for the house
%\item Increased members' involvement in the fraternity by assigning maintenance duties
%\end{list2}
%
%{\bf MIT International Students Association} \hfill \textit{Cambridge, MA}
%
%\vspace{-.4cm}
%\textbf {\emph{President}} \hfill {\bf May 2008 - May 2009}\\  
%\vspace*{-.17in}
%\begin{list2}
%\item Organized the MIT-Harvard-Wellesley Annual International Cruise and study breaks 
%\item Designed the setup and decorations of the MIT International Fair 2008 
%\item Improved awareness of international cultures in the MIT community
%\end{list2}

\section{\sc Professional Activities \& Service}
%\vspace*{-.17in}
\begin{list2}
\item[] American Physical Society member
\item[] American Astronomical Society member
%\item Romanian Student Association treasurer (2007-2010)
%\item MIT International Student Orientation coordinator (2008, 2009)
\end{list2}

\section{\sc Skills}
%\vspace*{-.17in}
\begin{list2}
\item[] Languages: Fluent in Romanian, English and Spanish, Conversant in German, Basics in French
\item[] Computer: Python, Mathematica, Matlab, LaTeX, C++, ROOT, Mac OS, Windows 2000/XP/Vista, Microsoft Office, Corel, Database Desktop
\end{list2}

%\section{\sc Honors and Awards}
%%\vspace*{-.17in}
%\begin{list2}
%\item Selected in the 20-person Romanian representative team for the International Physics Olympiad
%\item Awards in the Physics National Olympiad, Math and Chemistry Municipal Olympiads (2002-2005)
%\end{list2}

%\section{\sc Other Interests/Hobbies}
%\begin{list2}
%\item Dancing, swimming, shopping, fashion, reading, foreign languages, traveling
%\end{list2}





%\vspace{-.4cm}
%{\em Research Assistant} \hfill {\bf August, 2010 - present}\\  
%\vspace*{-.17in}
%\begin{list2}
%\item TBD1
%\item TBD2
%\item TBD3
%\end{list2}
%%\vspace{-.1cm}
%{\em NSF VIGRE Teaching Fellow} \hfill {\bf January - May, 2001}\\
%Head teaching assistant.   
%Duties included  shared administrative responsibilities with faculty
%instructor, fielding of all student inquiries, and oversight of
%graduate student teaching assistants and graders.
%\vspace*{.05in}  
%\begin{list2}
%\item 36-217 Probability Theory and Random Processes, Spring 2001.
%\end{list2}
%
%%\vspace{-.1cm}
%{\em Teaching Assistant} \hfill {\bf August, 2001  - present}\\
%Duties at various times have included 
%office hours and leading weekly computer lab exercises.
%
%
%
%\section{\sc Publications}
%Paciorek, C.J., J.S. Risbey, V. Ventura, and R.D.Rosen. 2002. Multiple indices of Northern Hemisphere Cyclone
%Activity, Winters 1949-1999. Journal of Climate 15:1573-1590.
%
%Paciorek, C.J., R. Condit, S.P. Hubbell, and R.B. Foster.  2000.
%The demographics of resprouting in tree and shrub species of a moist
%tropical forest.  Journal of Ecology 88:765-777.
%
%Paciorek, C.J., B.R. Moyer, R.A. Levin, and S.L. Halpern.  1995.
%Pollen consumption by hummingbird flower mite {\it Proctolaelaps
%  kirmsei} and possible fitness effects on {\it Hamelia patens}.
%Biotropica 27:258-262.  (author order determined by lot)     
%
%\section{\sc Papers in preparation}
%
%Ventura, V., C.J. Paciorek, and J.S. Risbey.  Controlling the proportion of falsely-rejected hypotheses when conducting multiple tests with geophysical data.
%
%Ickes, K., C.J. Paciorek, and S. Thomas.  Effects of wild pigs on
%forest demographic processes in Malaysia.
%
%\section{\sc Conference Presentations}
%Paciorek, C.J., J.S. Risbey, V. Ventura, and R.D.Rosen.  2001.  Changes in Northern Hemisphere winter storm activity (1949-1999) based
%on a comparison of cyclone indices.  8th International Meeting on
%Statistical Climatology, Luneberg, Germany, March, 2001.
%
%Paciorek, C.J. and R. Rosenfeld.  2000.  Minimum classification error
%training in exponential language models.  2000 Spring Transcription
%Workshop, College Park, Maryland.
%\vspace*{-.25in}  
%\begin{verbatim}http://www.nist.gov/speech/publications/tw00/html/abstract.htm#cp1-50\end{verbatim}
%
%\section{\sc Professional Experience}
%{\bf Bureau of Transportation Statistics, U.S. Department of
%  Transportation}, Washington, District of Columbia USA
%
%\vspace{-.3cm}
%{\em Summer researcher} \hfill {\bf May, 2000 - August, 2000}\\
%Carried out several consulting projects, including modelling of
%injuries to cadavers in crash test experiments, analysis of airline
%delay data, and advice on analysis of airline economics data.
%
%{\bf Abt Associates}, Bethesda, Maryland USA
%
%\vspace{-.3cm}
%{\em Associate Programmer Analyst and Research Assistant} \hfill {\bf
%  October, 1994 - August, 1996}\\
%Researcher and computer model developer for U.S. EPA Regulatory Impact
%Analysis of Section 403 Lead Paint Hazard Rule.  Other projects
%included database analysis, literature reviews, and cost-benefit analysis.
%
%\section{\sc Computer Skills} 
%\begin{list2}
%\item Statistical Packages:  R, S-Plus, BUGS; some experience
%  with SAS; extensive use of C and Fortran statistical libraries.
%\item Languages:  C++, Perl, Pascal, some use of Unix shell scripts,
%  MPI parallel processing library.
%\item Applications: Generic Mapping Tools (GMT) - Unix mapping software, \LaTeX, common Windows
%  database, spreadsheet, and presentation software
%\item Algorithms: Experience programming Markov Chain Monte Carlo
%  simulations of Bayesian posterior distributions
%\item Operating Systems:  Unix/Linux, Windows.\\ 
%\end{list2}

\end{resume}
\end{document}




